\documentclass[a4paper]{article}
\usepackage[utf8x]{inputenc}
\usepackage[T1,T2A]{fontenc}
\usepackage[russian]{babel}
\usepackage{hyperref}
\usepackage{indentfirst}
\usepackage{listings}
\usepackage{color}
\usepackage{here}
\usepackage{array}
\usepackage{multirow}
\usepackage{graphicx}
\usepackage{caption}
\graphicspath{{graphics/}}
\usepackage[left=2cm,right=2cm,
top=2cm,bottom=2cm,bindingoffset=0cm]{geometry}
\usepackage{listings}
\lstset{ %
	extendedchars=\true,
	keepspaces=true,
	language=bash,					% choose the language of the code
	basicstyle=\footnotesize,		% the size of the fonts that are used for the code
	numbers=left,					% where to put the line-numbers
	numberstyle=\footnotesize,		% the size of the fonts that are used for the line-numbers
	stepnumber=1,					% the step between two line-numbers. If it is 1 each line will be numbered
	numbersep=5pt,					% how far the line-numbers are from the code
	backgroundcolor=\color{white},	% choose the background color. You must add \usepackage{color}
	showspaces=false				% show spaces adding particular underscores
	showstringspaces=false,			% underline spaces within strings
	showtabs=false,					% show tabs within strings adding particular underscores
	frame=single,           		% adds a frame around the code
	tabsize=2,						% sets default tabsize to 2 spaces
	captionpos=b,					% sets the caption-position to bottom
	breaklines=true,				% sets automatic line breaking
	breakatwhitespace=false,		% sets if automatic breaks should only happen at whitespace
	escapeinside={\%*}{*)},			% if you want to add a comment within your code
	postbreak=\raisebox{0ex}[0ex][0ex]{\ensuremath{\color{red}\hookrightarrow\space}}
}

\begin{document}	% начало документа

\begin{titlepage}	% начало титульной страницы

	\begin{center}		% выравнивание по центру

		\large Санкт-Петербургский Политехнический Университет Петра Великого\\
		\large Институт компьютерных наук и технологий \\
		\large Кафедра компьютерных систем и программных технологий\\[6cm]
		% название института, затем отступ 6см
		
		\huge Телекоммуникационные технологии\\[0.5cm] % название работы, затем отступ 0,5см
		\large Отчет по лабораторной работе №8 \\[0.2cm]
		\large\textbf{"Модель телекоммуникационного канала"}\\[5cm]

	\end{center}


	\begin{flushright} % выравнивание по правому краю
		\begin{minipage}{0.25\textwidth} % врезка в половину ширины текста
			\begin{flushleft} % выровнять её содержимое по левому краю

				\large\textbf{Работу выполнила:}\\
				\large Власова А.В.\\
				\large {Группа:} 33501/4\\
				
				\large \textbf{Преподаватель:}\\
				\large Богач Н.В.\

			\end{flushleft}
		\end{minipage}
	\end{flushright}
	
	\vfill % заполнить всё доступное ниже пространство

	\begin{center}
	\large Санкт-Петербург\\
	\large \the\year % вывести дату
	\end{center} % закончить выравнивание по центру

\thispagestyle{empty} % не нумеровать страницу
\end{titlepage} % конец титульной страницы

\vfill % заполнить всё доступное ниже пространство

\section{Постановка задачи}
	Пакетный сигнал длительностью 200 мкс состоит из 64 бит полезной информации и 8 нулевых tail-бит. В нулевом 16-битном слове пакета передается ID, в первом - период излучения в мс, во втором – сквозной номер пакета, в третьем - контрольная сумма (CRC-16). На передающей стороне пакет сформированный таким образом проходит следующие этапы
	обработки:
	\begin{itemize}
		\item  Помехоустойчивое кодирование сверточным кодом с образующими полиномами 753, 561( octal ) и кодовым ограничением 9. На выходе кодера количество бит становится равным 144.
		\item Перемежение бит. Количество бит на этом этапе остается неизменным
		\item Модуляция символов. На этом этапе пакет из 144 полученных с выхода перемежителя бит разбивается на 24 символа из 6 бит. Генерируется таблица функций Уолша длиной 64 бита. Каждый 6-битный символ заменяется последовательностью Уолша, номер которой равен значению данных 6-ти бит. Т.о. на выходе модулятора получается 24 * 64 = 1536 знаковых символов.
		\item Прямое расширение спектра. Полученная последовательность из 1536 символов периодически умножается с учетом знака на ПСП длиной 511 символов. Далее к началу сформированного символьного пакета прикрепляется немодулированная ПСП. Т.о. символьная длина становится равной 1747. Далее полученные символы модулируются методом BPSK .
	\end{itemize}
	Задача: по имеющейся записи сигнала из эфира и коду модели передатчика создать модель приемника, в которой найти позицию начала пакета и, выполнив операции демодуляции, деперемежения и декодирования, получить передаваемые параметры: ID, период, и номер пакета. Известно, что ID = 4, период 100 мс, номер пакета 373. Запись сделана с передискретизацией 2, т.е. одному BPSK символу соответствуют 2 лежащих друг за другом отсчета в файле. Запись сделана на нулевой частоте и представляет из себя последовательность 32-х битных комплексных	отсчетов, где младшие 16 бит вещественная часть, старшие 16 бит – мнимая часть. Ниже приведена таблица перемежения и последовательность ПСП.

\section{Теоретический раздел}
Опорная псевдослучайная последовательность (ПСП) используется при синхронизации записи сигналов, выполняемой, если параметры шума в канале неизвестны. При демодуляции и одновременном сужении спектра принятого сигнала используется обратное преобразование Уолша-Адамара. При синхронизации и при сужении спектра выбирается максимальный по абсолютному значению элемент строки матрицы результатов, который указывает на начало пакета (при синхронизации) или на бинарный номер строки матрицы Уолша (при сужении спектра и демодуляции).
\section{Ход работы}
Получим сигнал от передатчика, выделим вещественную и мнимую части.
\captionof{lstlisting}{}
\lstinputlisting[firstline=50, lastline=55]{../lab8.m}

Проведем демодуляцию сигнала, построим матрицу Уолша.
\captionof{lstlisting}{}
\lstinputlisting[firstline=57, lastline=77]{../lab8.m}

Переведем полученный результат в двоичный код. 
\captionof{lstlisting}{}
\lstinputlisting[firstline=79, lastline=91]{../lab8.m}

Проведем деперемежение и декодирование.
\captionof{lstlisting}{}
\lstinputlisting[firstline=93, lastline=99]{../lab8.m}

\section{Выводы}
В ходе выполнения лабораторной работы разработана модель приемника данных. Для получения передаваемого сообщения приемник выполняет действия, обратные действиям передатчика: демодуляцию, деперемежение и декодирование полученных данных. Для того чтобы передача информации происходила без ошибок, на приемнике используется синхронизация записи сигнала по ПСП.
\end{document}



